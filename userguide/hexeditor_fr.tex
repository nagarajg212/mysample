\section{Éditeur Hexadécimal}\label{sec:hexeditor}

Comment ouvrir un fichier via HexEditor dans \codeblocks.

\begin{enumerate}
\item \menu{Fichier, Ouvrir avec HexEditor}
\item Menu contextuel du navigateur de projet (\menu{Ouvrir avec,Hex editor}
\item Sélectionnez l'onglet Fichier dans le panneau de gestion. En sélectionnant un fichier dans le Gestionnaire de fichiers et en exécutant le menu contextuel \menu{Ouvrir avec Hex Editor}, le fichier s'ouvre dans HexEditor.
\end{enumerate}

Répartition des fenêtres :

À gauche la Vue de HexEditor et à droite l'affichage sous forme de chaînes de caractères

Ligne du haut :
Position actuelle (valeur en décimal/hex) et pourcentage (rapport entre la position actuelle du curseur et le fichier complet).

Boutons:

Fonctions de recherche

Bouton Aller à : Sauter à une position absolue. Format décimal ou hexadécimal. Saut relatif vers l'avant ou vers l'arrière en spécifiant le signe.

Chercher: Recherchez des motifs hexadécimaux dans la vue HexEditor ou des chaînes de caractères dans la vue d'aperçu de fichier.

Configuration du nombre de colonnes :
Exactement, Multiple de, Puissance de

Mode d'affichage :
Hexa, Binaire

Octets :
Sélectionnez le nombre d'octets à afficher par colonne.

Choix d'Endianess :
BE: Big Endian
LE: Little Endian

Valeur Prévisualisée :
Ajoute une vue supplémentaire dans HexEditor. Pour une valeur sélectionnée dans HexEditor, la valeur est également affichée sous forme de Word, Dword, Float, Double.

Entrée d'expression :
Permet d'effectuer une opération arithmétique sur une valeur dans HexEditor. Le résultat de l'opération est affiché dans la marge de droite.

Calc:
Testeur d'Expression

Édition d'un fichier dans HexEditor :

Commandes d'historique Annuler (Undo) et Refaire (Redo).

Autre exemple, Déplacer le curseur dans la vue des chaînes de caractères :
Insérer des espaces avec la touche Insérer.
Supprimer des caractères en appuyant sur la touche Suppr.

En saisissant un texte, le contenu existant est écrasé sous la forme d'une chaîne de caractères.

En saisissant des chiffres dans la vue d'HexEditor, les valeurs sont écrasées et l'aperçu est mis à jour.

